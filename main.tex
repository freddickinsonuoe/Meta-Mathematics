% TODO: go through the entire preamble and sort what's not needed!

\documentclass[12pt]{article}

% Character encoding
\usepackage[utf8]{inputenc}
\usepackage[a4paper]{geometry}

% maths packages
\usepackage{amsmath}
\usepackage{amssymb}  
\usepackage{amsthm}
\usepackage{gensymb}
\usepackage{lipsum}

% https://ctan.org/pkg/enumitem?lang=en
\usepackage{enumitem}
\setlist[enumerate, 1]{label={(\roman*)}}


% for images, code,
\usepackage{graphicx}  
\usepackage{listings}  
\usepackage{tabto} 
\usepackage{hyperref}
\usepackage{soul}

% various particular use packages
\usepackage{bm}
\usepackage{etoc}
\usepackage[normalem]{ulem}

% upload .sty file separately
\usepackage{simples-matrices}
\usepackage{minted}

% hyperlink setup
\usepackage{hyperref} % added for hyperlinks$
\hypersetup{colorlinks=true,
            linkcolor=cyan,    
            filecolor=magenta,
            urlcolor=orange,
            pdftitle={We love maths.}, 
            pdfpagemode=FullScreen}

% only for contents
\setcounter{tocdepth}{2}

\usepackage[framemethod=TikZ]{mdframed}
\usepackage{./thmboxes_v2}

\usepackage{csquotes}
\usepackage[
    natbib=true,
    style=numeric,
    sorting=none
]{biblatex}
\addbibresource{bibliography.bib}


% personal command overrides
\newcommand{\Log}{\text{Log}}
\newcommand{\Arg}{\text{Arg}}
\newcommand{\Mat}{\text{Mat}}
\newcommand{\can}{\text{can}}
\newcommand{\umlauto}{\text{ö}}
\newcommand{\acutee}{\text{é}}
\newcommand{\gravea}{\text{à}}
\newcommand{\hot}{\mathcal{H}.\mathcal{O}.\mathcal{T}}
\newcommand{\Res}{\text{Res}}
\newcommand{\Int}{\text{Int}}
\newcommand{\andd}{~\text{and}~}
\newcommand*\conj[1]{\overline{#1}}

\newcommand{\area}{\text{area}}
\newcommand{\dt}{\mathrm{d}t}

\newcommand{\dz}{\mathrm{d}z}
\newcommand{\dx}{\mathrm{d}x}
\newcommand{\R}{\mathbb{R}}
\newcommand{\C}{\mathbb{C}}
\newcommand{\Z}{\mathbb{Z}}
\newcommand{\N}{\mathbb{N}}
\newcommand{\F}{\mathbb{F}}
\newcommand{\Q}{\mathbb{Q}}
\newcommand{\ncr}{\matrice[1]{n, r}}



% START HERE
\title{Meta-Mathematics: General Notes}
\author{}
\date{}

\begin{document}
\maketitle

\tableofcontents
\vspace{1cm}
{\Large This document is for notes and collaboration purposes only.}
\pagebreak

\section{Project Information}
\href{https://webapps.maths.ed.ac.uk/student_choices/project_info.php?id=UG1429}{This link is for the project information on the UoE site.}\\
Recommended reading includes:
\begin{enumerate}
    \item Al Cuoco. Meta-Problems in Mathematics. The College Mathematics Journal, Vol. 31, No. 5 (Nov., 2000), pp. 373-378.
    \item Aaron E. Naiman, Generating parametrized linear systems for teaching linear algebra. International Journal of Mathematical Education in Science and Technology, DOI: 10.1080/0020739X.2022.2114107
\end{enumerate}

\begin{quote}
    \textit{This is a project in mathematics education, undertaking a survey of "meta-mathematical problems". That is, the mathematics needed to set feasible mathematics questions. For example, a teacher wanting to write exercises on the Pythagorean Theorem will seek right-angled triangles with integer sides, e.g. 3,4,5 triangles. This project will survey papers in this field, and collect together key mathematics which teachers regularly need in order to set problems. Note, this project will require some elementary number theory.}
\end{quote}

\subsection{Using Git}
This project is linked to a \href{https://github.com/freddickinsonuoe/Meta-Mathematics}{a GitHub repository}. You can sync your GitHub account with overleaf to make the experience quite simple. To avoid conflict
\begin{enumerate}[itemsep=-2mm]
    \item before starting to make changes, ensure you are up to date with the latest version. See ``Menu - GitHub - Pull Changes";
    \item when you want to make changes, push them to the repository. Again see ``Menu - GitHub - Push Changes"
\end{enumerate}

\subsection{BibTex \& Paper Access}
We use \textsc{BibTex} to store information about our references. Most papers are stored in the overleaf file tree and are named ``author/year". To reference a paper existing in the bibliography (bibliography.bib) use
\begin{verbatim}
    \cite{authorYear}
\end{verbatim}
that will produce \cite{green1976} (in the case of Green, 1976).

\subsection{Rough Plan}
Looking to do a ``storytelling" project on meta-mathematics. Possible sections could be
\begin{enumerate}[itemsep=-2mm]
    \item what is meta mathematics (where can we apply),
    \item its history,
    \item common themes (e.g pythagorean triples),
    \item exploring a theme in more detail (cubics)
    \item exploring more theme(s) in detail \textbf{if applicable},
    \item closing.
\end{enumerate}

\noindent The ``in more detail" could include: how we create nice cubics, cardano's formula, applying this to past papers, ...\\\\
(\textbf{Cubics}) Good papers include Johnson (2015), Jeffrey (2004), Buddenhagen (1992).\\
(\textbf{General)} See Steele, Cuoco, Shirali.

\subsection{Common Themes}
For now just a basic list;
\begin{enumerate}[itemsep=-2mm]
    \item pythagoras' theorem,
    \item diophantine equations,
    \item Gaussian integers,
    \item Eisenstein integers,
    
\end{enumerate}

\subsection{Glossary of Terms}
\begin{enumerate}[itemsep=-2mm]
    \item \textsc{Maple} - ``math software that combines the world's most powerful math engine with an interface that makes it extremely easy to analyze, explore, visualize, and solve mathematical problems."
\end{enumerate}


\pagebreak 

\section{Reading Material}

\subsection{AI Cuoco (2000): Meta-Problems in Mathematics}
``Problems that come out nice allow students to concentrate on important ideas rather than messy calculations". We call the idea of making up a nice problem a ``meta-problem". Algebra and number theory are key for the creation of a lot of meta-problems - Cuoco explores a few examples.

\subsubsection{Pythagorean Triples}
Suppose you're a secondary school teacher looking to teach the Pythagoras theorem. Nice numbers allow students to focus on the problem at hand, rather than getting bogged down in complicated arithmetic.\\\\
Recall that any scalar multiplication of a Pythagorean triple gives a new triple;
\begin{equation*}
    (da)^2 + (db)^2 = d^2(a^2 + b^2) = d^2(c^2) = (dc)^2
\end{equation*}
so the bigger question is looking for \textbf{primitive} Pythagorean triples. One method for this is using Euclid's formula (as seen before in number theory);
\begin{thm}[Euclid's Formula for PPT's]{}{}
If $m$ and $n$ are two \textbf{odd integers} such that $m > n$ then the triple $(a, b, c)$ formed
\begin{equation*}
    a = mn,\quad b = \frac{m^2 - n^2}{2},\quad c = \frac{m^2 + n^2}{2}
\end{equation*}
is indeed a Pythagorean triple. This triple is primitive if and only if $m$ and $n$ are coprime; in fact, every PPT arises from coprime odd $m > n > 0$.
\end{thm}
\textit{*note - there is a technicality in the changing of $a$ and $b$.}\\\\
We can use computers to help us generate any Pythagorean triple suitable for class problems. See the next page for an example script.\\\\
\hl{Including the script in \LaTeX{} is not pretty so I've deleted it for now, but I have it on my Jupyter Notebook.}

\subsection{Johnson (2015): Nice Cubics}
Gives a detailed explanation and algorithm for producing nice cubics; there's a big link to Eisenstein triples (and Eisenstein integers) which is similar to material we've seen before in number theory.

\subsubsection{Gaussian Integers vs Eisenstein Integers}
The Gaussian integers, as seen before in number theory, is the ring $\Z[i]$ s.t
\begin{equation*}
    \Z[i] = a + bi \qquad \text{where}~a,b\in \Z.
\end{equation*}
This has a big link to \textit{Pythagorean triples.} Recall the norm of some Gaussian integer $z \in \Z[i]$ is given
\begin{equation*}
    N(z) = ||a+bi||_2 = (a+bi)(a-bi) = a^2 + b^2.
\end{equation*}
As $a$ and $b$ are both integers then so is the norm. We 

\subsection{Mansfield (2017) \& Robson (2001): Plimpton 322}
The Plimpton 322 is a Babylonian clay tablet with four rows and fifteen columns of various numbers. Robson (2001) provides a ``re-assessment" of what the content of the tablet means. Mansfield does similar, quoting
\begin{quote}
    \textit{``we propose that P322 is a different kind of trigonometric table which lists right triangles with long side 1, exact short side $\beta$ and exact diagonal $\beta$ – in place of the approximations $\sin(\theta)$ and $\cos(\theta)$".}
\end{quote}
They are both quite historical papers with less mathematical content than others.

\subsection{Steele (2003): Setting Linear Algebra Problems}
\textit{``I was disappointed by the level of arithmetic competence of my students, and annoyed to find that problems were being set where this lack of competence meant one could never really see whether students understood the material and techniques or not"} - the same rationale as Cuoco in setting problems that allow students to focus on the problem at hand.\\\\
Provides methods for coming up with nice problems for various linear algebra topics. Occasionally makes use of \textsc{Maple}.

\subsubsection{Vectors of Integer Norm}
In $\R^2$ this is simply finding Pythagorean triples. He also makes use of Lagrange's theorem;
\begin{thm}[Lagrange's Theorem]{}{}
Any integer can be decomposed into the sum of at most four squares.
\end{thm}
So this is useful for finding vectors $\bm{x} \in \R^4$ with integer norm. \hl{He does not give any detail into how we may decompose these integers.}

\subsubsection{Matrices of a Given Determinant}
Recall the LU factorisation from ILA/NLA/Honours Algebra
\begin{dfn}[\textbf{LU} Factorisation]{}{}
An \textbf{LU factorisation} of some $\bm{A} \in \R^{n\times n}$ is $\bm{A} = \bm{LU}$ where
\begin{enumerate}[itemsep=-2mm]
    \item $\bm{L} \in \R^{n \times n}$ is unit lower triangular,
    \item $\bm{U} \in \R^{n \times n}$ is non-singular upper triangular.
\end{enumerate}
\end{dfn}
Unit lower triangular means entries on the diagonal are one. A better decomposition involves a permutation matrix $P$ as \textit{any} matrix can be decomposed into a permutation matrix $P$ and $L$, $U$ as above.\\\\
The most useful way to use the $PLU$ decomposition is to work backwards and create matrices with determinants $\pm1$. This is because of the following theorem. \textit{He does not explicitly go through how to create such matrices but does comment on it.}
\begin{thm}[Inverse of Matrices with Unit Determinants]{}{}
Let $\mathbf{A}$ be a square integer matrix. Then $\mathbf{A}^{-1}$ is an integer matrix iff $\det(\mathbf{A}) = \pm1$.
\end{thm}
\textit{He gives a proof of this theorem.}

\subsubsection{Further Chapters}
He goes through a lot more topics in similar detail. The chapters are
\begin{enumerate}[itemsep=-2mm]
    \item matrices with a given kernel,
    \item leontieff input-output matrices,
    \item orthogonal matrices,
    \item gram-schmidt and qr decomposition,
    \item least-squares problems,
    \item householder matrices,
    \item matrices with a given jordan normal form,
    \item orthogonal matrices in $\R^3$,
    \item symmetric matrices with given eigenvalues,
    \item matrices with easily calculated exponential.
\end{enumerate}

\subsection{Naiman (2022) - Generating Parametrized Linear Systems}
...  for Teaching Linear Algebra.\\\\
Again follows the rationale of creating problems that allow students to focus on the  task at hand instead of arithmetic. This paper focuses on:
\begin{quote}
\textit{For what values of the input parameter $k$, does the following linear system,
    $\mathbf{ax} = b$, have: 
    \begin{enumerate}[itemsep=-2mm]
        \item no solutions,
        \item a unique solution (and what it is),
        \item      an infinite number of solutions (and what they are).
    \end{enumerate}}
\end{quote}
They go through their process and the challenges faced before providing a full algorithm for the problem.

\subsection{Convertito (2016): Building the Biggest Box}
... Three Factor Polynomials and a Diophantine Equation.\\\\
Discusses optimising the volume of a box by cutting squares out of the corner of a square/rectangular piece of cardboard. The problem is treated by optimizing a third degree polynomial and solves by finding all three-factor polynomials whose first derivatives have m-rational roots.\\\\
The paper is majority proofs of theorems considering a more general calculus problem that in turn leads to a more general Diophantine equation than the one used specifically for the box problem.\\\\
The box problem specifically finds all the solutions of the Diophantine equation.
\[a^2 + 3b^2 = c^2, \quad a, b, c \in \mathbb{N}.\]\\

\subsection{Buddenhagen (1992) - Nice Cubic Polynomials and More}
The paper has the intent of providing a systematic and unified presentation for finding ``nice" cubics. Here nice means the roots and critical points are rational numbers with small denominators.\\\\
They go through, using various different parts of mathematics to provide a solution for this problem in quite good depth.

\subsection{Jeffrey (2004): Not seeing the roots for the branches: multivalued functions in computer algebra}
This paper examines the validity of Cardano's Method by comparing its results to mathematical software \textsc{Maple} and \textsc{Mathematica}. Indeed, depending on the definition of the cube root used, Cardano's Method can produce an answer equivalent to the one of the software. For example, when Cardano's Method is applied to a reduced cubic of the form $y^3 + 3py - 2q = 0$, the roots $y_1, y_2, y_3$ are given as follows:
\begin{align*}
    &y_1 = s_1 + s_2\\
    &y_2 = - \frac{1}{2} (s_1+s_2) + i \frac{\sqrt{3}}{2}(s_1-s_2)\\
    &y_3 = - \frac{1}{2} (s_1+s_2) - i \frac{\sqrt{3}}{2}(s_1-s_2)
\end{align*}
with
\begin{align*}
    &s_1 = (q+\sqrt{q^2 + p^3} )^\frac{1}{3}\\
    &s_2 = (q-\sqrt{q^2 + p^3} )^\frac{1}{3}\\
    &p = -s_1 s_2
\end{align*}
These results are incorrect when considering the principal value or the real branch as the definition of the cube root. However, the results are correct when the cube root is defined as a value from the set $z^\frac{1}{3}$.


\subsection{Green (1976): The Historical Development of Complex Numbers}
This provides a nice example of Cardano's Method in action and explains its limitations through Cardano's historical context. He notes the lack of information around complex numbers at that time, as Cardano was the first to use $\sqrt{-}$ in a calculation, and the prevalence of strictly positive constants in cubic equations. This led to Cardano describing his method as "refined as it is useless" when he mistakenly converted $\sqrt{-15}$ to $- \sqrt{15}$.

\subsection{Srikanth \& Sudheer (2020): A note on the solutions of cubic equations of state in low temperature region}
This paper shows Cardano's Method's impact on real life applications, here the calculation of the compressibility factor of cubic equations of state used in the chemical and petroleum industries. It highlights the method's limits when compared to real results and compares it to Lagrange's which works better in this case. 

\subsection{Sangwin (2018): Textbook accounts of the rules of indices with rational exponents}
This highlights another flaw of Cardano's formula: the confusion around how to expand and factorize $\sqrt{-a} \sqrt{-b}$ to apply the formula. Indeed, as pointed out in the paper, $\sqrt{-a} \sqrt{-b}$ could be written as
\[\sqrt{-a} \sqrt{-b} = \sqrt{-a \times -b} = \sqrt{ab} \]
or as
\[\sqrt{-a} \sqrt{-b} = (\sqrt{-1} \times \sqrt{a}) \times (\sqrt{-1} \times \sqrt{b}) = (\sqrt{-1})^2 \sqrt{a} \sqrt{b} = -\sqrt{ab} \]
This aligns well with Green (1976) which highlighted the misconceptions about complex numbers of Cardano's time.

\subsection{Zelator (2011): Integer roots of quadratic and cubic polynomials with integer coefficients}
Uses elementary number theory to analyse ways to make a cubic equation $x^3 + bx^2 + cx + d$ have integer roots and coefficients. Finds five 'special cases' and the required conditions:\\\\

Special Case 1: The number 0 is a root of c(x); so that d=0 requires d=0 and $b^2-4c=k^2$; k a nonnegative integer $r_1 = \frac{-b + k}{2}, \quad r_2 = \frac{-(b + k)}{2}, \quad r_3 = 0$\\

Special Case 2: The number 1 is a root of c(x) requires $b+c+d=-1$ and $(b+1)^2+4d = k^2$; k a nonnegative integer $r_1 = \frac{-(b + 1) + k}{2}, \quad r_2 = \frac{-(b + 1 + k)}{2}, \quad r_3 = 1$\\

Special Case 3: The number -1 is a root of c(x) requires $b-c+d=1 and (b-1)^2-4d=k^2$; k a nonnegative integer $r_1 = \frac{b - 1 + k}{2}, \quad r_2 = \frac{b - 1 - k}{2}, \quad r_3 = -1$\\

Special Case 4: c(x) has a triple (i.e. multiplicity 3) integer root requires $c=\frac{b^2}{3} and d= \frac{b^3}{27}$ The integer root is $r=-b/3$\\

Special Case 5: c(x) has a double integer root and another integer root has four different possible ways of being true.

\section{Creating Nice Cubics}
This gives an explanation of the introduction to Johnson's paper and then explores how we can generate these so-called \textit{Eisenstein triples}.

\subsection{Derivation of the Diophantine Equation}
We first consider some cubic. We want it to have \textit{integer roots} and a \textit{rational turning point.} Up to suitable translation and scaling, we can consider a monic polynomial with one root at zero i.e
\begin{equation*}
    f(x) = x(x-a)(x-b)
\end{equation*}
where $a$ and $b$ are our integer roots. Expanding this we find
\begin{align*}
    f(x) &= x(x^2 - (a + b)x + ab)\\
    &= x^3 - (a+b)x^2 + abx
\end{align*}
and thus the derivative to consider is
\begin{equation*}
    f'(x) = 3x^2 - 2(a+b)x + ab.
\end{equation*}
Of course the values at the \textit{turning point} are when this derivative is equal to zero. We use the quadratic formula (or completing the square, essentially analogous) to find the solutions to this equation
\begin{equation*}
    x = \frac{2(a+b) \pm \sqrt{(-2(a+b))^2-12ab}}{6} = \frac{(a+b) \pm \sqrt{(a+b)^2 - 3ab}}{3}.
\end{equation*}
We want the turning point to be \textit{rational} i.e for some integer $c \in \Z$
\begin{equation}\label{eq:diophantine_eq1}
    (a+b)^2 - 3ab = c^2 \implies a^2 - ab + b^2 = c^2.
\end{equation}
This is the \textit{Diophantine equation} that will henceforth be referenced. Recall we have all $a, b, c$ integers.

\subsection{What are Eisenstein Integers?}
The notion of \textit{Pythagorean triples} is familiar; the triple $(a, b, c)$ is a Pythagorean triple if $a^2 + b^2 = c^2$. Geometrically this links to a right angled triangle. We have seen in number theory that these are linked to Gaussian integers. Indeed, for some Gaussian integer $z \in \Z[i]$ we can introduce the norm
\begin{equation*}
    N(z) = z\conj{z} = (a + bi)(a + b\conj{i}) = (a+bi)(a-bi) =  a^2 + b^2.
\end{equation*}
So a Pythagorean triple $(a, b, c)$ can be found by taking the norm of any Gaussian integer; that is, if $z = a + bi \in \Z[i]$ then $N(z) = c^2$.\\\\
Eisenstein triples are similar. Recall the Cosine rule (also known as the law of cosines)
\begin{equation*}
    c^2 = a^2 + b^2 - 2ab\cos(C)
\end{equation*}
and suppose we are looking for triangles with one angle of $60$ degrees. We know $\cos(60) = \frac{1}{2}$ and thus the equation becomes
\begin{equation*}
    c^2 = a^2 - ab + b^2
\end{equation*}
and triples that satisfy such equation (look familiar?) are called \textit{Eisenstein triples}.\\\\
We now introduce Eisenstein integers. These are to Eisenstein triples what Gaussian integers are to Pythagorean triples. Eisenstein integers are the ring
\begin{equation*}
    \Z[\omega] = \{a + b\omega : a, b \in \Z\}
\end{equation*}
where $\omega$ is a cube root of unity. Being a cube root of unity gives $\omega$ a lot of properties, so we explore it in its entirety. By definition, a cube root of unity is a root to the equation
\begin{equation*}
    (\omega-1)(\omega^2 + \omega + 1) = 0
\end{equation*}
and thus we can assert our first property $\omega^2 = -(1 + \omega)$. We often represent (the complex values of) $\omega$ 
\begin{equation*}
    \omega = \frac{-1 \pm i\sqrt{3}}{2} = e^{\frac{2\pi i }{3}}
\end{equation*}
and it's easy to show that $\conj{\omega} = \omega^2$ (consider the transformation on an Argand diagram).\\\\
Circling back to our Diophantine equation; how are these linked? We've mentioned that they are analogous to that of Gaussian integers to the Pythagorean triples and indeed, if we consider the norm of some $z \in \Z[\omega]$
\begin{equation*}
    N(z) = z\conj{z} = (a+b\omega)(a+b\conj{\omega}) = \dots = a^2 - ab + b^2
\end{equation*}
where the algebraic simplification is an exercise in using the properties of $\omega$ outlined above. So, as we saw before, we can generate an Eisenstein triple $(a, b, c)$ by considering the norm of an Eisenstein integer $N(a + b\omega)$ \dots\\\\
\dots and we've seen how this is done by Johnson \cite{johnson2011} and Read \cite{read2006}. They have different approaches and formulas but one is given (in this case by Read)
\begin{quote}
    \textit{Let $p, q$ be positive integers such that $p$ and $q$ are coprime and $q$ is not divisible by $3$. Then the triple
    \begin{align*}
        a &= 2pq + q^2,\\
        b &= 3p^2 + 4pq + q^2,\\
        c &= 3p^2 + 3pq + q^2
    \end{align*}
    form a (primitive) triangle such that one of the angles is 60 degrees}
\end{quote}
\dots i.e an Eisenstein triple. We can test this with $p = 1, q = 2$ (for example) and see that we indeed have a solution $a = 8, b = 15, c= 13$ where the cubic $x(x-8)(x-15)$ is ``nice". This is not exhaustive; there are other formulae (at least for Read, Johnson is slightly more complicated) needed.

\subsection{Can We Use It?}
If we can find a new way to generate triples/solve this Diophantine equation (perhaps using Eisenstein triples, perhaps using complex numbers) then we can use it in our project. If not, Chris probably won't approve - it's ``nothing new". At the very least it can be used in common themes - this is the second time Eisenstein triples have come up now, they're also seen in Cuoco's paper \autocite{cuoco2000} and applied to the context of finding nice cubics.


\section{Past Paper Questions}

\subsection{A-Level Further Maths Cubic Questions}
\textbf{Pearson Edexcel Level 3 GCE Further Mathematics
Advanced PAPER 1: Core Pure Mathematics 1}\\\\
\textbf{Question 1 June 2022:}\\\\
Consider the function 
\[f(z) = z^3 + az + 52\]
where \( a \) is a real constant. Given that \( 2 - 3i \) is a root of the equation \( f(z) = 0 \).
\begin{enumerate}
    \item[(a)] Write down the other complex root.
    \item[(b)] Hence,
    \begin{enumerate}
        \item[(i)] Solve completely \( f(z) = 0 \).
        \item[(ii)] Determine the value of \( a \).
    \end{enumerate}
    \item[(c)] Show all the roots of the equation \( f(z) = 0 \) on a single Argand diagram.\\\\
\end{enumerate}

\textbf{Question 3 June 2021:}\\\\
The cubic equation
\[ax^3 + bx^2 - 19x - b = 0\]
where \( a \) and \( b \) are constants, has roots \( \alpha \), \( \beta \), and \( \gamma \).

The cubic equation
\[w^3 - 9w^2 - 97w + c = 0\]
where \( c \) is a constant, has roots \( (4\alpha - 1) \), \( (4\beta - 1) \), and \( (4\gamma - 1) \).
Without solving either cubic equation, determine the value of \( a \), the value of \( b \), and the value of \( c \).\\\\

\textbf{Question 1 June 2020:}
Let 
\[f(z) = 3z^3 + pz^2 + 57z + q\]
where \( p \) and \( q \) are real constants. Given that \( 3 - 2\sqrt{2}i \) is a root of the equation \( f(z) = 0 \):
\begin{enumerate}
    \item[(a)] Show all the roots of \( f(z) = 0 \) on a single Argand diagram.    
    \item[(b)] Find the value of \( p \) and the value of \( q \).\\\\
\end{enumerate}

\textbf{Question 1 June 2019 (4th power not 3rd):}
\[f(z) = z^4 + az^3 + bz^2 + cz + d\] where \( a \), \( b \), \( c \), and \( d \) are real constants.\\
Given that \( -1 + 2i \) and \( 3 - i \) are two roots of the equation \( f(z) = 0 \):

\begin{enumerate}
    \item[(a)] Show all the roots of \( f(z) = 0 \) on a single Argand diagram.
    \item[(b)] Find the values of \( a \), \( b \), \( c \), and \( d \).\\\\
\end{enumerate}

\textbf{Pearson Edexcel Level 3 GCE Further Mathematics
Advanced PAPER 2: Core Pure Mathematics 2}\\\\

\textbf{Question 6 June 2022:}\\
The cubic equation
\[4x^3 + px^2 - 14x + q = 0\]
where \( p \) and \( q \) are real positive constants, has roots \( \alpha \), \( \beta \), and \( \gamma \).

Given that \( \alpha^2 + \beta^2 + \gamma^2 = 16 \):

\begin{enumerate}
    \item[(a)] Show that \( p = 12 \).
    
    Given that 
    \[\frac{1}{\alpha} + \frac{1}{\beta} + \frac{1}{\gamma} = \frac{14}{\alpha \beta \gamma}\]
    \item[(b)] Determine the value of \( q \).
    
    Without solving the cubic equation,
    \item[(c)] Determine the value of \( (\alpha - 1)(\beta - 1)(\gamma - 1) \).
\end{enumerate}

\textbf{Question 2 June 2019:}\\
The roots of the equation 
\[x^3 - 2x^2 + 4x - 5 = 0\]
are \( p \), \( q \), and \( r \).

Without solving the equation, find the value of:
\begin{enumerate}
    \item[(i)] \( 2p + q^2 + 2r \)
    \item[(ii)] \( (p - 4)(q - 4)(r - 4) \)
    \item[(iii)] \( p^3 + q^3 + r^3 \)
\end{enumerate}



\subsection{A-Level Maths Cubic Questions}

\textbf{Pearson Edexcel Level 3 GCE Mathematics
Advanced PAPER 1: Pure Mathematics 1}\\\\
\textbf{Question 6 June 2022:}\\
A curve \( C \) has equation \( y = f(x) \), where \( f(x) \) is a cubic expression in \( x \). The curve:
\begin{itemize}
    \item passes through the origin,
    \item has a maximum turning point at \( (2, 8) \),
    \item has a minimum turning point at \( (6, 0) \).
\end{itemize}
\begin{enumerate}
    \item[(a)] Write down the set of values of \( x \) for which \( f'(x) < 0 \).    
    \item[(b)] The line with equation \( y = k \), where \( k \) is a constant, intersects \( C \) at only one point. Find the set of values of \( k \), giving your answer in set notation.    
    \item[(c)] Find the equation of \( C \). You may leave your answer in factorised form.
\end{enumerate}


\textbf{Pearson Edexcel Level 3 GCE Mathematics
Advanced PAPER 2: Pure Mathematics 2}\\\\
\textbf{Question 1 June 2023:}\\
Let \[f(x) = x^3 + 2x^2 - 8x + 5.\]
\begin{enumerate}
    \item[(a)] Find \( f''(x) \). 
    \item[(b)] 
    \begin{enumerate}
        \item[(i)] Solve \( f''(x) = 0 \).
        \item[(ii)] Hence, find the range of values of \( x \) for which \( f(x) \) is concave.\\\\
    \end{enumerate}
\end{enumerate}

\textbf{Question 5 June 2023:}\\
The curve \( C \) has equation \( y = f(x) \). The curve
\begin{itemize}
    \item passes through the point \( P(3, -10) \)
    \item has a turning point at \( P \)
\end{itemize}

Given that 
\[\frac{dy}{dx} = 2x^3 - 9x^2 + 5x + k\]
where \( k \) is a constant,
\begin{enumerate}
    \item[(a)] Show that \( k = 12 \).
    \item[(b)] Hence, find the coordinates of the point where \( C \) crosses the y-axis.\\\\
\end{enumerate}

\textbf{Question 7 June 2023:}\\
A curve has equation
\[x^3 + 2xy + 3y^2 = 47.\]
\begin{enumerate}
    \item[(a)] Find \( \frac{dy}{dx} \) in terms of \( x \) and \( y \). 
    The point \( P(-2, 5) \) lies on the curve.    
    \item[(b)] Find the equation of the normal to the curve at \( P \), giving your answer in the form \( ax + by + c = 0 \), where \( a \), \( b \), and \( c \) are integers to be found.\\\\
\end{enumerate}

\textbf{Question 5 November 2021:}\\
The curve \( C \) has equation 
\[y = 5x^4 - 24x^3 + 42x^2 - 32x + 11, \quad x \in \mathbb{R}\]
\begin{enumerate}
    \item[(a)] Find:
    \begin{enumerate}
        \item[(i)] \( \frac{dy}{dx} \)
        \item[(ii)] \( \frac{d^2y}{dx^2} \)
    \end{enumerate}  
    \item[(b)] Verify that:
    \begin{enumerate}
        \item[(i)] \( C \) has a stationary point at \( x = 1 \).
        \item[(ii)] Show that this stationary point is a point of inflection, giving reasons for your answer.
    \end{enumerate}    
\end{enumerate}

% End of document - with bibliography.
\pagebreak 


% End of document - with bibliography.
\pagebreak 

\nocite{*}
\printbibliography{}



\end{document}


