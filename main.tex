% TODO: go through the entire preamble and sort what's not needed!

\documentclass[12pt]{article}

% Character encoding
\usepackage[utf8]{inputenc}
\usepackage[a4paper]{geometry}

% maths packages
\usepackage{amsmath}
\usepackage{amssymb}  
\usepackage{amsthm}
\usepackage{gensymb}
\usepackage{lipsum}

% https://ctan.org/pkg/enumitem?lang=en
\usepackage{enumitem}
\setlist[enumerate, 1]{label={(\roman*)}}


% for images, code,
\usepackage{graphicx}  
\usepackage{listings}  
\usepackage{tabto} 
\usepackage{hyperref}
\usepackage{soul}

% various particular use packages
\usepackage{bm}
\usepackage{etoc}
\usepackage[normalem]{ulem}

% upload .sty file separately
\usepackage{simples-matrices}
\usepackage{minted}

% hyperlink setup
\usepackage{hyperref} % added for hyperlinks$
\hypersetup{colorlinks=true,
            linkcolor=cyan,    
            filecolor=magenta,
            urlcolor=orange,
            pdftitle={We love maths.}, 
            pdfpagemode=FullScreen}

% only for contents
\setcounter{tocdepth}{2}

\usepackage[framemethod=TikZ]{mdframed}
\usepackage{./thmboxes_v2}

\usepackage{csquotes}
\usepackage[
    natbib=true,
    style=numeric,
    sorting=none
]{biblatex}
\addbibresource{bibliography.bib}


% personal command overrides
\newcommand{\Log}{\text{Log}}
\newcommand{\Arg}{\text{Arg}}
\newcommand{\Mat}{\text{Mat}}
\newcommand{\can}{\text{can}}
\newcommand{\umlauto}{\text{ö}}
\newcommand{\acutee}{\text{é}}
\newcommand{\gravea}{\text{à}}
\newcommand{\hot}{\mathcal{H}.\mathcal{O}.\mathcal{T}}
\newcommand{\Res}{\text{Res}}
\newcommand{\Int}{\text{Int}}
\newcommand{\andd}{~\text{and}~}
\newcommand*\conj[1]{\overline{#1}}

\newcommand{\area}{\text{area}}
\newcommand{\dt}{\mathrm{d}t}

\newcommand{\dz}{\mathrm{d}z}
\newcommand{\dx}{\mathrm{d}x}
\newcommand{\R}{\mathbb{R}}
\newcommand{\C}{\mathbb{C}}
\newcommand{\Z}{\mathbb{Z}}
\newcommand{\N}{\mathbb{N}}
\newcommand{\F}{\mathbb{F}}
\newcommand{\Q}{\mathbb{Q}}
\newcommand{\ncr}{\matrice[1]{n, r}}



% START HERE
\title{Meta-Mathematics: General Notes}
\author{}
\date{}

\begin{document}
\maketitle

\tableofcontents
\vspace{1cm}
{\Large This document is for notes and collaboration purposes only.}
\pagebreak

\section{Project Information}
\href{https://webapps.maths.ed.ac.uk/student_choices/project_info.php?id=UG1429}{This link is for the project information on the UoE site.}\\
Recommended reading includes:
\begin{enumerate}
    \item Al Cuoco. Meta-Problems in Mathematics. The College Mathematics Journal, Vol. 31, No. 5 (Nov., 2000), pp. 373-378.
    \item Aaron E. Naiman, Generating parametrized linear systems for teaching linear algebra. International Journal of Mathematical Education in Science and Technology, DOI: 10.1080/0020739X.2022.2114107
\end{enumerate}

\begin{quote}
    \textit{This is a project in mathematics education, undertaking a survey of "meta-mathematical problems". That is, the mathematics needed to set feasible mathematics questions. For example, a teacher wanting to write exercises on the Pythagorean Theorem will seek right-angled triangles with integer sides, e.g. 3,4,5 triangles. This project will survey papers in this field, and collect together key mathematics which teachers regularly need in order to set problems. Note, this project will require some elementary number theory.}
\end{quote}

\subsection{Using Git}
This project is linked to a \href{https://github.com/freddickinsonuoe/Meta-Mathematics}{a GitHub repository}. You can sync your GitHub account with overleaf to make the experience quite simple. To avoid conflict
\begin{enumerate}[itemsep=-2mm]
    \item before starting to make changes, ensure you are up to date with the latest version. See ``Menu - GitHub - Pull Changes";
    \item when you want to make changes, push them to the repository. Again see ``Menu - GitHub - Push Changes"
\end{enumerate}

\subsection{BibTex}
Info on BibTex.

\subsection{Paper Access and Storing}
Info on how we will share and access relevant papers.

\pagebreak 

\section{Reading Material}

\subsection{AI Cuoco: Meta-Problems in Mathematics}
``Problems that come out nice allow students to concentrate on important ideas rather than messy calculations". We call the idea of making up a nice problem a ``meta-problem". Algebra and number theory are key for the creation of a lot of meta-problems - Cuoco explores a few examples.

\subsubsection{Pythagorean Triples}
Suppose you're a secondary school teacher looking to teach the Pythagoras theorem. Nice numbers allow students to focus on the problem at hand, rather than getting bogged down in complicated arithmetic.\\\\
Recall that any scalar multiplication of a Pythagorean triple gives a new triple;
\begin{equation*}
    (da)^2 + (db)^2 = d^2(a^2 + b^2) = d^2(c^2) = (dc)^2
\end{equation*}
so the bigger question is looking for \textbf{primitive} Pythagorean triples. One method for this is using Euclid's formula (as seen before in number theory);
\begin{thm}[Euclid's Formula for PPT's]{}{}
If $m$ and $n$ are two \textbf{odd integers} such that $m > n$ then the triple $(a, b, c)$ formed
\begin{equation*}
    a = mn,\quad b = \frac{m^2 - n^2}{2},\quad c = \frac{m^2 + n^2}{2}
\end{equation*}
is indeed a Pythagorean triple. This triple is primitive if and only if $m$ and $n$ are coprime; in fact, every PPT arises from coprime odd $m > n > 0$.
\end{thm}
\textit{*note - there is a technicality in the changing of $a$ and $b$.}\\\\
We can use computers to help us generate any Pythagorean triple suitable for class problems. See the next page for an example script.\\\\
\hl{Including the script in \LaTeX{} is kinda ugly so might delete later?}

\pagebreak 
\noindent \textbf{An example script to generate all Pythagorean triples $(a, b, c)$ with an upper bound on $c$.}

\begin{minted}{python}
def generate_pythagorean_triples(c_max):
    """
    Generates pythagorean triples with no component greater than c_max.
    Uses (a variant of) Euclid's formula.
    """

    # generates ppt's
    m, n = 3, 1
    pythag_triples = []

    while (m**2 + n**2 < c_max):
        # iterate through for n fixed, change m
        while (m**2 + n**2 < c_max):
            triple = calc_triple(m, n)
            pythag_triples.append(triple)

            m = m + 2

        # change n and reset m, resetting the process
        n = n + 2
        m = n + 2 


    # ppt's have been generated, so generate any scalar component too
    scalar_triples = []
    for ppt in pythag_triples:
        a, b, c = ppt
        k = 2

        while (c * k) < c_max:
            new_triple = (a * k, b * k, c * k)
            scalar_triples.append(new_triple)
            k = k + 1
            
    all_pythag_triples = pythag_triples + scalar_triples
    return all_pythag_triples
\end{minted}
We run the function with an upper bound of $50$ and verify that all outputs are Pythagorean triples.
\begin{minted}{python}
def calc_triple(m, n):
    """
    Calculates a Pythagorean triple using Euclid's formula.
    Assumes odd integers m > n
    """
    
    a = m * n
    b = int((m**2 - n**2) / 2)
    c = int((m**2 + n**2) / 2)

    pythag_triple = (a, b, c)
    return pythag_triple


all_triples = generate_pythagorean_triples(50)
for unchecked_triple in all_triples:
    a, b, c = unchecked_triple
    if (a**2 + b**2) != c**2:
        print("Something went wrong!")

# all_triples.sort()
all_triples

> [(3, 4, 5), (5, 12, 13), (15, 8, 17), (6, 8, 10), (9, 12, 15), 
(12, 16, 20),(15, 20, 25), (18, 24, 30), (21, 28, 35), (24, 32, 40), 
(27, 36, 45), (10, 24, 26), (15, 36, 39), (30, 16, 34)]


\end{minted}

\pagebreak 

\nocite{*}
\printbibliography{}





\end{document}
