% TODO: go through the entire preamble and sort what's not needed!

\documentclass[12pt]{article}

% Character encoding
\usepackage[utf8]{inputenc}
\usepackage[a4paper]{geometry}

% maths packages
\usepackage{amsmath}
\usepackage{amssymb}  
\usepackage{amsthm}
\usepackage{gensymb}
\usepackage{lipsum}

% https://ctan.org/pkg/enumitem?lang=en
\usepackage{enumitem}
\setlist[enumerate, 1]{label={(\roman*)}}


% for images, code,
\usepackage{graphicx}  
\usepackage{listings}  
\usepackage{tabto} 
\usepackage{hyperref}
\usepackage{soul}

% various particular use packages
\usepackage{bm}
\usepackage{etoc}
\usepackage[normalem]{ulem}

% upload .sty file separately
\usepackage{simples-matrices}
\usepackage{minted}

% hyperlink setup
\usepackage{hyperref} % added for hyperlinks$
\hypersetup{colorlinks=true,
            linkcolor=cyan,    
            filecolor=magenta,
            urlcolor=orange,
            pdftitle={We love maths.}, 
            pdfpagemode=FullScreen}

% only for contents
\setcounter{tocdepth}{2}

\usepackage[framemethod=TikZ]{mdframed}
\usepackage{./thmboxes_v2}

\usepackage{csquotes}
\usepackage[
    natbib=true,
    style=numeric,
    sorting=none
]{biblatex}
\addbibresource{bibliography.bib}


% personal command overrides
\newcommand{\Log}{\text{Log}}
\newcommand{\Arg}{\text{Arg}}
\newcommand{\Mat}{\text{Mat}}
\newcommand{\can}{\text{can}}
\newcommand{\umlauto}{\text{ö}}
\newcommand{\acutee}{\text{é}}
\newcommand{\gravea}{\text{à}}
\newcommand{\hot}{\mathcal{H}.\mathcal{O}.\mathcal{T}}
\newcommand{\Res}{\text{Res}}
\newcommand{\Int}{\text{Int}}
\newcommand{\andd}{~\text{and}~}
\newcommand*\conj[1]{\overline{#1}}

\newcommand{\area}{\text{area}}
\newcommand{\dt}{\mathrm{d}t}

\newcommand{\dz}{\mathrm{d}z}
\newcommand{\dx}{\mathrm{d}x}
\newcommand{\R}{\mathbb{R}}
\newcommand{\C}{\mathbb{C}}
\newcommand{\Z}{\mathbb{Z}}
\newcommand{\N}{\mathbb{N}}
\newcommand{\F}{\mathbb{F}}
\newcommand{\Q}{\mathbb{Q}}
\newcommand{\ncr}{\matrice[1]{n, r}}



% START HERE
\title{Meta-Mathematics: General Notes}
\author{}
\date{}

\begin{document}
\maketitle

\tableofcontents
\vspace{1cm}
{\Large This document is for notes and collaboration purposes only.}
\pagebreak

\section{Project Information}
\href{https://webapps.maths.ed.ac.uk/student_choices/project_info.php?id=UG1429}{This link is for the project information on the UoE site.}\\
Recommended reading includes:
\begin{enumerate}
    \item Al Cuoco. Meta-Problems in Mathematics. The College Mathematics Journal, Vol. 31, No. 5 (Nov., 2000), pp. 373-378.
    \item Aaron E. Naiman, Generating parametrized linear systems for teaching linear algebra. International Journal of Mathematical Education in Science and Technology, DOI: 10.1080/0020739X.2022.2114107
\end{enumerate}

\begin{quote}
    \textit{This is a project in mathematics education, undertaking a survey of "meta-mathematical problems". That is, the mathematics needed to set feasible mathematics questions. For example, a teacher wanting to write exercises on the Pythagorean Theorem will seek right-angled triangles with integer sides, e.g. 3,4,5 triangles. This project will survey papers in this field, and collect together key mathematics which teachers regularly need in order to set problems. Note, this project will require some elementary number theory.}
\end{quote}

\subsection{Using Git}
This project is linked to a \href{https://github.com/freddickinsonuoe/Meta-Mathematics}{a GitHub repository}. You can sync your GitHub account with overleaf to make the experience quite simple. To avoid conflict
\begin{enumerate}[itemsep=-2mm]
    \item before starting to make changes, ensure you are up to date with the latest version. See ``Menu - GitHub - Pull Changes";
    \item when you want to make changes, push them to the repository. Again see ``Menu - GitHub - Push Changes"
\end{enumerate}

\subsection{BibTex \& Paper Access}
We use \textsc{BibTex} to store information about our references. Most papers are stored in the overleaf file tree and are named ``author/year". To reference a paper existing in the bibliography (bibliography.bib) use
\begin{verbatim}
    \cite{authorYear}
\end{verbatim}
that will produce \cite{green1976} (in the case of Green, 1976).

\subsection{Rough Plan}
Looking to do a ``storytelling" project on meta-mathematics. Possible sections could be
\begin{enumerate}[itemsep=-2mm]
    \item what is meta mathematics (where can we apply),
    \item its history,
    \item common themes (e.g pythagorean triples),
    \item exploring a theme in more detail (cubics)
    \item exploring more theme(s) in detail \textbf{if applicable},
    \item closing.
\end{enumerate}

\noindent The ``in more detail" could include: how we create nice cubics, cardano's formula, applying this to past papers, ...\\\\
(\textbf{Cubics}) Good papers include Johnson (2015), Jeffrey (2004), Buddenhagen (1992).\\
(\textbf{General)} See Steele, Cuoco, Shirali.

\subsection{Common Themes}
For now just a basic list;
\begin{enumerate}[itemsep=-2mm]
    \item pythagoras' theorem,
    \item diophantine equations,
    \item Gaussian integers,
    \item Eisenstein integers,
    
\end{enumerate}

\subsection{Glossary of Terms}
\begin{enumerate}[itemsep=-2mm]
    \item \textsc{Maple} - ``math software that combines the world's most powerful math engine with an interface that makes it extremely easy to analyze, explore, visualize, and solve mathematical problems."
    \item \textsc{Mathematica} - also sometimes called \textsc{Wolfram Mathematica} ``software system with built-in libraries for several areas of technical computing that allow machine learning, symbolic computation, optimization, plotting functions and various types of data, implementation of algorithms, interfacing with programs written in other programming languages, etc"
\end{enumerate}


\pagebreak 

\section{Reading Material}

\subsection{AI Cuoco (2000): Meta-Problems in Mathematics}
``Problems that come out nice allow students to concentrate on important ideas rather than messy calculations". We call the idea of making up a nice problem a ``meta-problem". Algebra and number theory are key for the creation of a lot of meta-problems - Cuoco explores a few examples.

\subsubsection{Pythagorean Triples}
Suppose you're a secondary school teacher looking to teach the Pythagoras theorem. Nice numbers allow students to focus on the problem at hand, rather than getting bogged down in complicated arithmetic.\\\\
Recall that any scalar multiplication of a Pythagorean triple gives a new triple;
\begin{equation*}
    (da)^2 + (db)^2 = d^2(a^2 + b^2) = d^2(c^2) = (dc)^2
\end{equation*}
so the bigger question is looking for \textbf{primitive} Pythagorean triples. One method for this is using Euclid's formula (as seen before in number theory);
\begin{thm}[Euclid's Formula for PPT's]{}{}
If $m$ and $n$ are two \textbf{odd integers} such that $m > n$ then the triple $(a, b, c)$ formed
\begin{equation*}
    a = mn,\quad b = \frac{m^2 - n^2}{2},\quad c = \frac{m^2 + n^2}{2}
\end{equation*}
is indeed a Pythagorean triple. This triple is primitive if and only if $m$ and $n$ are coprime; in fact, every PPT arises from coprime odd $m > n > 0$.
\end{thm}
\textit{*note - there is a technicality in the changing of $a$ and $b$.}\\\\
We can use computers to help us generate any Pythagorean triple suitable for class problems. See the next page for an example script.\\\\
\hl{Including the script in \LaTeX{} is not pretty so I've deleted it for now, but I have it on my Jupyter Notebook.}

\subsection{Johnson (2015): Nice Cubics}
Gives a detailed explanation and algorithm for producing nice cubics; there's a big link to Eisenstein triples (and Eisenstein integers) which is similar to material we've seen before in number theory.

\subsubsection{Gaussian Integers vs Eisenstein Integers}
The Gaussian integers, as seen before in number theory, is the ring $\Z[i]$ s.t
\begin{equation*}
    \Z[i] = a + bi \qquad \text{where}~a,b\in \Z.
\end{equation*}
This has a big link to \textit{Pythagorean triples.} Recall the norm of some Gaussian integer $z \in \Z[i]$ is given
\begin{equation*}
    N(z) = ||a+bi||_2 = (a+bi)(a-bi) = a^2 + b^2.
\end{equation*}
As $a$ and $b$ are both integers then so is the norm. We 

\subsection{Mansfield (2017) \& Robson (2001): Plimpton 322}
The Plimpton 322 is a Babylonian clay tablet with four rows and fifteen columns of various numbers. Robson (2001) provides a ``re-assessment" of what the content of the tablet means. Mansfield does similar, quoting
\begin{quote}
    \textit{``we propose that P322 is a different kind of trigonometric table which lists right triangles with long side 1, exact short side $\beta$ and exact diagonal $\beta$ – in place of the approximations $\sin(\theta)$ and $\cos(\theta)$".}
\end{quote}
They are both quite historical papers with less mathematical content than others.

\subsection{Steele (2003): Setting Linear Algebra Problems}
\textit{``I was disappointed by the level of arithmetic competence of my students, and annoyed to find that problems were being set where this lack of competence meant one could never really see whether students understood the material and techniques or not"} - the same rationale as Cuoco in setting problems that allow students to focus on the problem at hand.\\\\
Provides methods for coming up with nice problems for various linear algebra topics. Occasionally makes use of \textsc{Maple}.

\subsubsection{Vectors of Integer Norm}
In $\R^2$ this is simply finding Pythagorean triples. He also makes use of Lagrange's theorem;
\begin{thm}[Lagrange's Theorem]{}{}
Any integer can be decomposed into the sum of at most four squares.
\end{thm}
So this is useful for finding vectors $\bm{x} \in \R^4$ with integer norm. \hl{He does not give any detail into how we may decompose these integers.}

\subsubsection{Matrices of a Given Determinant}
Recall the LU factorisation from ILA/NLA/Honours Algebra
\begin{dfn}[\textbf{LU} Factorisation]{}{}
An \textbf{LU factorisation} of some $\bm{A} \in \R^{n\times n}$ is $\bm{A} = \bm{LU}$ where
\begin{enumerate}[itemsep=-2mm]
    \item $\bm{L} \in \R^{n \times n}$ is unit lower triangular,
    \item $\bm{U} \in \R^{n \times n}$ is non-singular upper triangular.
\end{enumerate}
\end{dfn}
Unit lower triangular means entries on the diagonal are one. A better decomposition involves a permutation matrix $P$ as \textit{any} matrix can be decomposed into a permutation matrix $P$ and $L$, $U$ as above.\\\\
The most useful way to use the $PLU$ decomposition is to work backwards and create matrices with determinants $\pm1$. This is because of the following theorem. \textit{He does not explicitly go through how to create such matrices but does comment on it.}
\begin{thm}[Inverse of Matrices with Unit Determinants]{}{}
Let $\mathbf{A}$ be a square integer matrix. Then $\mathbf{A}^{-1}$ is an integer matrix iff $\det(\mathbf{A}) = \pm1$.
\end{thm}
\textit{He gives a proof of this theorem.}

\subsubsection{Further Chapters}
He goes through a lot more topics in similar detail. The chapters are
\begin{enumerate}[itemsep=-2mm]
    \item matrices with a given kernel,
    \item leontieff input-output matrices,
    \item orthogonal matrices,
    \item gram-schmidt and qr decomposition,
    \item least-squares problems,
    \item householder matrices,
    \item matrices with a given jordan normal form,
    \item orthogonal matrices in $\R^3$,
    \item symmetric matrices with given eigenvalues,
    \item matrices with easily calculated exponential.
\end{enumerate}

\subsection{Naiman (2022) - Generating Parametrized Linear Systems}
...  for Teaching Linear Algebra.\\\\
Again follows the rationale of creating problems that allow students to focus on the  task at hand instead of arithmetic. This paper focuses on:
\begin{quote}
\textit{For what values of the input parameter $k$, does the following linear system,
    $\mathbf{ax} = b$, have: 
    \begin{enumerate}[itemsep=-2mm]
        \item no solutions,
        \item a unique solution (and what it is),
        \item      an infinite number of solutions (and what they are).
    \end{enumerate}}
\end{quote}
They go through their process and the challenges faced before providing a full algorithm for the problem.

\subsection{Convertito (2016): Building the Biggest Box}
... Three Factor Polynomials and a Diophantine Equation.\\\\
Discusses optimising the volume of a box by cutting squares out of the corner of a square/rectangular piece of cardboard. The problem is treated by optimizing a third degree polynomial and solves by finding all three-factor polynomials whose first derivatives have m-rational roots.\\\\
The paper is majority proofs of theorems considering a more general calculus problem that in turn leads to a more general Diophantine equation than the one used specifically for the box problem.\\\\
The box problem specifically finds all the solutions of the Diophantine equation.
\[a^2 + 3b^2 = c^2, \quad a, b, c \in \mathbb{N}.\]\\

\subsection{Buddenhagen (1992) - Nice Cubic Polynomials and More}
The paper has the intent of providing a systematic and unified presentation for finding ``nice" cubics. Here nice means the roots and critical points are rational numbers with small denominators.\\\\
They go through, using various different parts of mathematics to provide a solution for this problem in quite good depth.

\subsection{Gordon (2012) - Properties of Eisenstein Triples}
As the title suggests, this points out nice Number Theory properties of Eisenstein triples and gives their proofs. The three most interesting and useable ones in my opinion are the following.
\begin{thm}{}{} %% I don't get why the numbering doesn't follow the rest but it's not a big issue
    If $(a,b,c)$ is a primitive Eisenstein triple, then $c$ is neither a multiple of $2$ nor $3$. More generally, the only prime factors of $c$ are primes of the form $6k + 1$.
\end{thm}
\begin{thm}
If $(a,b,c)$ is an Eisenstein triple, then
\begin{enumerate}
    \item one of the numbers $a$, $b$, or $b-a$ is divisible by $8$
    \item one of the numbers $a$, $b$, or $b-a$ is divisible by $3$
    \item one of the numbers $a$, $b$, or $b-a$ is divisible by $5$
    \item one of the numbers $a$, $b$, $c$, or $b-a$ is divisible by $7$
\end{enumerate}
\end{thm}
\begin{thm}{}{}
    If $(a,b,c)$ is an Eisenstein triple, then $2c > 2b - a$ and $2c > a + b$.
\end{thm}
\begin{thm}{}{}
    Suppose that $a$, $b$, and $c$ are positive integers.
    \begin{enumerate}
        \item If $(a,b,c)$ is an Eisenstein triple with $a<b-a$, then $(a,b-a,c)$ is a $120$° triple.
        \item If $(a,b,c)$ is a $120$° triple, then $(a, a+b, c)$ and $(b,a+b,c)$ are Eisenstein triples.
    \end{enumerate}
\end{thm}
For context, a $120$° triple is to an Eisenstein triple what a $120$° angle is to a $60$° angle. It might be interesting to focus on them too, but there may be less information on them. We say $(a,b,c)$ is a $120$° triple if $0<a<b<c$ and $a^2 + ab + b^2 = c^2$. Figure 2 of the paper nicely shows the relation between the two types of triples. Maybe using a combination of both for meta-mathematical purposes could be nice?

\subsection{Jeffrey (2004): Not seeing the roots for the branches: multivalued functions in computer algebra}
This paper examines the validity of Cardano's Method by comparing its results to mathematical software \textsc{Maple} and \textsc{Mathematica}. Indeed, depending on the definition of the cube root used, Cardano's Method can produce an answer equivalent to the one of the software. For example, when Cardano's Method is applied to a reduced cubic of the form $y^3 + 3py - 2q = 0$, the roots $y_1, y_2, y_3$ are given as follows:
\begin{align*}
    &y_1 = s_1 + s_2\\
    &y_2 = - \frac{1}{2} (s_1+s_2) + i \frac{\sqrt{3}}{2}(s_1-s_2)\\
    &y_3 = - \frac{1}{2} (s_1+s_2) - i \frac{\sqrt{3}}{2}(s_1-s_2)
\end{align*}
with
\begin{align*}
    &s_1 = (q+\sqrt{q^2 + p^3} )^\frac{1}{3}\\
    &s_2 = (q-\sqrt{q^2 + p^3} )^\frac{1}{3}\\
    &p = -s_1 s_2
\end{align*}
These results are incorrect when considering the principal value or the real branch as the definition of the cube root. However, the results are correct when the cube root is defined as a value from the set $z^\frac{1}{3}$.


\subsection{Green (1976): The Historical Development of Complex Numbers}
This provides a nice example of Cardano's Method in action and explains its limitations through Cardano's historical context. He notes the lack of information around complex numbers at that time, as Cardano was the first to use $\sqrt{-}$ in a calculation, and the prevalence of strictly positive constants in cubic equations. This led to Cardano describing his method as "refined as it is useless" when he mistakenly converted $\sqrt{-15}$ to $- \sqrt{15}$.

\subsection{Srikanth \& Sudheer (2020): A note on the solutions of cubic equations of state in low temperature region}
This paper shows Cardano's Method's impact on real life applications, here the calculation of the compressibility factor of cubic equations of state used in the chemical and petroleum industries. It highlights the method's limits when compared to real results and compares it to Lagrange's which works better in this case. 

\subsection{Sangwin (2018): Textbook accounts of the rules of indices with rational exponents}
This highlights another flaw of Cardano's formula: the confusion around how to expand and factorize $\sqrt{-a} \sqrt{-b}$ to apply the formula. Indeed, as pointed out in the paper, $\sqrt{-a} \sqrt{-b}$ could be written as
\[\sqrt{-a} \sqrt{-b} = \sqrt{-a \times -b} = \sqrt{ab} \]
or as
\[\sqrt{-a} \sqrt{-b} = (\sqrt{-1} \times \sqrt{a}) \times (\sqrt{-1} \times \sqrt{b}) = (\sqrt{-1})^2 \sqrt{a} \sqrt{b} = -\sqrt{ab} \]
This aligns well with Green (1976) which highlighted the misconceptions about complex numbers of Cardano's time.

\subsection{Zelator (2011): Integer roots of quadratic and cubic polynomials with integer coefficients}
Uses elementary number theory to analyse ways to make a cubic equation $x^3 + bx^2 + cx + d$ have integer roots and coefficients. Finds five 'special cases' and the required conditions:\\\\

Special Case 1: The number 0 is a root of c(x); so that d=0 requires d=0 and $b^2-4c=k^2$; k a nonnegative integer $r_1 = \frac{-b + k}{2}, \quad r_2 = \frac{-(b + k)}{2}, \quad r_3 = 0$\\

Special Case 2: The number 1 is a root of c(x) requires $b+c+d=-1$ and $(b+1)^2+4d = k^2$; k a nonnegative integer $r_1 = \frac{-(b + 1) + k}{2}, \quad r_2 = \frac{-(b + 1 + k)}{2}, \quad r_3 = 1$\\

Special Case 3: The number -1 is a root of c(x) requires $b-c+d=1 and (b-1)^2-4d=k^2$; k a nonnegative integer $r_1 = \frac{b - 1 + k}{2}, \quad r_2 = \frac{b - 1 - k}{2}, \quad r_3 = -1$\\

Special Case 4: c(x) has a triple (i.e. multiplicity 3) integer root requires $c=\frac{b^2}{3} and d= \frac{b^3}{27}$ The integer root is $r=-b/3$\\

Special Case 5: c(x) has a double integer root and another integer root has four different possible ways of being true.

\pagebreak 

\section{Exploring The Cubic}

\subsection{Elementary Results}
Elementary results, sometimes with proofs.

\subsubsection{Critical Points and Inflection Points}
A \textit{cubic function} is a degree three polynomial $f(x) = ax^3 + bx^2 + cx + d$ where $a, b, c, d$ are constants with $a \neq 0$. The function has three roots with \textit{at least one} real root. The derivative of a cubic function is a quadratic function.\\\\
The graph of a cubic function always has a single \textit{inflection point}. It may also have two \textit{critical points}, and if it does not, we say the graph is monotonic.
\begin{dfn}[Inflection Point]
The inflection point is a point on a smooth curve where the curvature changes sign; this is where a function changes from being concave to convex. This point can be found by solving $f''(x) = 0$.
\end{dfn}
The definition of critical points and monotonicity are not written here.
\begin{thm}[Cubic Functions and Inflection Points]
A cubic function always has a single, unique inflection point.
\end{thm}
\begin{proof}
Consider some cubic function $ax^3 + bx^2 + cx + d$ where $a, b, c, d \in \R, a \neq 0$. Taking the first and second derivatives,
\begin{align*}
    f'(x) &= 3ax^2 + 2bx + cd,\\
    f''(x) &= 6ax + 2b.
\end{align*}
As $a \neq 0$, the second derivative $f''(x)$ is a linear function in $x$ and thus has one unique solution to $f''(x) = 0$. It follows that there exists a single, unique point of inflection for the cubic function.
\end{proof}
\textit{Remark.} A ``cubic equation" is defined as a function
\begin{equation*}
    ax^3 + bx^2 + cx + d = 0,\qquad a \neq 0.
\end{equation*}

\subsubsection{Roots of Odd Degree Polynomials}
We use the notion that a real number is a particular kind of complex number $z = a + bi$ where $b = 0$. First we recall two theorems.
\begin{rcl}[The Fundamental Theorem of Algebra]
If $P(z)$ is a non-zero, single variable degree $n$ polynomial with complex coefficients then it has, counted with multiplicity, exactly $n$ complex roots.
\end{rcl}
\textit{Remark.} This is an alternative definition for the fundamental theorem of algebra. It is equivalent to saying that the complex numbers $\mathbb{C}$ are algebraically closed.
\begin{rcl}[The Intermediate Value Theorem]
If $f$ is a continuous function whose domain contains $[a, b]$, then it takes on any given value between $f(a)$ and $f(b)$ at some point in $[a, b]$.
\end{rcl}
\begin{thm}[Roots of Odd Degree Polynomials]
Any real-valued polynomial with odd degree have at least one real root.
\end{thm}
\begin{proof}
    \textit{(By the IVT)} Consider a polynomial $P(x)$ with real coefficients and degree $n$, where $n$ is odd i.e
    \begin{equation*}
        P(x) = a_0 + a_1x + a_2x^2 + \dots + a_nx^n = \sum_{k = 0}^n \alpha_kx^k.
    \end{equation*}
    Assume $a_n > 0$. Then as $x \rightarrow \infty$, $P(x) \rightarrow \infty$; and similarly, as $x \rightarrow -\infty$, $P(x) \rightarrow -\infty$. Then, by the intermediate value theorem, there is some point $y \in \R$ such that $P(y) = 0$.
\end{proof}
\textit{Remark.} Informal proof with some missing parts. The case of $a_n < 0$ is analogous.
\begin{proof}
    (\textit{By the FTA)} Consider a polynomial $P(z)$ with real coefficients and degree $n$, where $n$ is odd. By the fundamental theorem of algebra, $P(z)$ has, counted with multiplicity, exactly $n$ complex roots. We consider the set of roots $\{z_j\}_{j=1}^n$ and note, by the theorem on complex conjugates,
    \begin{equation*}
        z \in \{z_j\}_{j=1}^n \implies \conj{z} \in \{z_j\}_{j=1}^n.
    \end{equation*}
    However, as $n$ is odd, there must exist at least one root $z$ such that $z = \bar{z}$ and thus such $z \in \R$, as required.
\end{proof}

\subsection{Depressed Cubics}
\begin{dfn}[Depressed Cubics]
A \textit{depressed cubic} is a cubic with no quadratic term; that is, if $P(x) = ax^3 
+ bx^2 + cx + d$ then a depressed cubic is the case where $b = 0$.
\end{dfn}
\textit{Remark.} I'm pretty sure we also have $a = 1$.
\begin{thm}[Cubics Can Be Sad Too]
All cubic functions can be depressed; that is, any cubic can be reduced, by a change of variable, such that it is a depressed cubic.
\end{thm}
\textit{Remark.} We first give intuitive explanations before proceeding with a more formal proof.\\\\
\textit{Intuitive Approach.} Consider any cubic, and now consider transforming such cubic such that its point of inflection is about $x = 0$. We've seen that
\begin{equation*}
    P(x) = ax^3 + bx^2 + cx + d \implies P''(x) = 6ax + 2b
\end{equation*}
and, given the point of inflection happens when $x = -\frac{b}{3a}$, if the point of inflection is about $x = 0$ then necessarily $b = 0$.\\\\
\textit{Intuitive Approach \#2.} As before, we can transform any cubic such that its point of inflection is about $x = 0$. Therefore it is an odd function; $P(x) = -P(x)$. Thus, necessarily $b = 0$ as the cubic
\begin{equation*}
    P(x) = ax^3 + bx^2 + cx + d
\end{equation*}
is odd if and only if $b = 0$.
\begin{proof}
    \textit{(Formal proof).} Canno be bothered.
\end{proof}

\noindent \textbf{The change of variable.} The change of variable in question is
\begin{equation*}
    x = t - \frac{b}{3a}
\end{equation*}
which should align with our intuitive understanding. If the point of inflection is at $x = -\frac{b}{3a}$ then by translating the graph across by such value, the inflection point is now about $x = 0$. Indeed, if we consider the depressed cubic $t^3 + pt + q = 0$, then
\begin{equation*}
    t = x + \frac{b}{3a}, \quad p =\frac{3ac - b^2}{3a^2}, \quad q = \frac{2b^3 - 9abc + 27a^2d}{27a^3}
\end{equation*}
and again, the roots are linked
\begin{equation*}
    x_k = t_k - \frac{b}{3a},\qquad k = 1,2,3.
\end{equation*}

\subsubsection{Solving Depressed Cubics via Linear Methods}
See \href{https://www.desmos.com/calculator/xkcdxf7sy2}{this Desmos demonstration} for reference.\\\\
Any depressed cubic can be solved by simultaneously solving
\begin{align*}
    ay + bx + c &= 0,\\
    y &= x^3.
\end{align*}
This comes quite easily from considering the substitution $y = x^3$ into the form of a depressed cubic.

\subsection{The Discriminant}
The discriminant of a polynomial is a function of its coefficients - we know that for a quadratic, the discriminant is $b^2 - 4ac$. Formally, we say the that the discriminant is nonzero if and only if the polynomial has no squares.
\begin{dfn}[Discriminant of a Cubic]
The discriminant $\mathbf{\Delta}$ of a cubic function $ax^3 + bx^2 + cx + d$ is given
\begin{equation*}
    \mathbf{\Delta} = 18abcd - 4b^3d + b^2c^2 - 4ac^3 - 27a^2d^2.
\end{equation*}
For a depressed cubic $t^3 + pt + q$ this reduces down to
\begin{equation*}
    \mathbf{\Delta} = -(4p^3 + 27q^2)
\end{equation*}
and if the roots $r_1, r_2, r_3$ are known,
\begin{equation*}
    \mathbf{\Delta} = a^4(r_1 - r_2)^2(r_1 - r_3)^2(r_2 - r_3)^2.
\end{equation*}
\end{dfn}
If the coefficients of a cubic are real and the discriminant is not zero, then
\begin{enumerate}[itemsep=-2mm]
    \item if $\mathbf{\Delta} > 0$ then the cubic has three distinct real roots,
    \item if $\mathbf{\Delta} < 0$ then the cubic has one real root and two complex conjugate roots.
\end{enumerate}
If the discriminant is zero, then it has a multiple root. Furthermore, if it has real coefficients, then all of its roots are real. This can be easily shown using the definition of the discriminant involving the roots.\\
The determinant of a cubic is associated to the determinant of its quadratic derivative. Indeed, for $\Delta_2 = (2b)^2 - 4(3a)c = 4(b^2-3ac)$ the determinant of the derivative of the cubic, we have 
$$27a^2 \Delta \leq \frac{1}{16} (\Delta_2)^3$$

\pagebreak 

\section{Creating Nice Cubics}
This gives an explanation of the introduction to Johnson's paper and then explores how we can generate these so-called \textit{Eisenstein triples}.

\subsection{What is 'Nice'?}
Before discovering the ways to create so called 'nice' cubics, we first must define what we mean by nice. The obvious place to start is with integers; a cubic with integer roots and/or integer turning points would generally be considered a nice cubic. However, does that mean we consider $$14688x^3+499392x^2-13938912x-233333568$$ a nice cubic just because it has roots at integer values $x=13, x=-26$ and $x=-47$? Clearly, there needs to be further detail into the definition of nice.\\\\
Let us say that any integer with an absolute value less than or equal to $50$ is considered 'nice'. This term may apply to: coefficients in the cubic equation; roots of the equation; turning points of the cubic and in some cases the real and imaginary parts of a Gaussian integer. This seems like a reasonable number to call the cutoff for a nice integer.\\\\
We can take this newly defined definition of a nice integer to allow us to define a nice rational number. Let a nice rational number be a number with co-prime integers for its numerator and denominator, each with an absolute value less than or equal to $50$.\\\\
Defining what is meant by a nice irrational number gets trickier. There are obvious answers, such as $\pi$ or $e$, or $\sqrt{2}$, but how do we define the set of all 'nice' irrational numbers? As this paper will be mainly focusing on cubic equations, irrational numbers shouldn't be used as our nice outcomes very often, but we will define them anyway.\\
We may define an irrational number involving $\pi$ as 'nice' if it is of the form $\frac{a\pi}{b}$ where $a$ and $b$ are co-prime integers both of absolute value less than $15$.\\
Similarly, any irrational square root may be considered nice if the number in the square root sign is positive and less than $15$.


\subsection{Derivation of the Diophantine Equation}
We first consider some cubic. We want it to have \textit{integer roots} and a \textit{rational turning point.} Up to suitable translation and scaling, we can consider a monic polynomial with one root at zero i.e
\begin{equation*}
    f(x) = x(x-a)(x-b)
\end{equation*}
where $a$ and $b$ are our integer roots. Expanding this we find
\begin{align*}
    f(x) &= x(x^2 - (a + b)x + ab)\\
    &= x^3 - (a+b)x^2 + abx
\end{align*}
and thus the derivative to consider is
\begin{equation*}
    f'(x) = 3x^2 - 2(a+b)x + ab.
\end{equation*}
Of course the values at the \textit{turning point} are when this derivative is equal to zero. We use the quadratic formula (or completing the square, essentially analogous) to find the solutions to this equation
\begin{equation*}
    x = \frac{2(a+b) \pm \sqrt{(-2(a+b))^2-12ab}}{6} = \frac{(a+b) \pm \sqrt{(a+b)^2 - 3ab}}{3}.
\end{equation*}
We want the turning point to be \textit{rational} i.e for some integer $c \in \Z$
\begin{equation}\label{eq:diophantine_eq1}
    (a+b)^2 - 3ab = c^2 \implies a^2 - ab + b^2 = c^2.
\end{equation}
This is the \textit{Diophantine equation} that will henceforth be referenced. Recall we have all $a, b, c$ integers.

\subsection{What are Eisenstein Integers?}
The notion of \textit{Pythagorean triples} is familiar; the triple $(a, b, c)$ is a Pythagorean triple if $a^2 + b^2 = c^2$. Geometrically this links to a right angled triangle. We have seen in number theory that these are linked to Gaussian integers. Indeed, for some Gaussian integer $z \in \Z[i]$ we can introduce the norm
\begin{equation*}
    N(z) = z\conj{z} = (a + bi)(a + b\conj{i}) = (a+bi)(a-bi) =  a^2 + b^2.
\end{equation*}
So a Pythagorean triple $(a, b, c)$ can be found by taking the norm of any Gaussian integer; that is, if $z = a + bi \in \Z[i]$ then $N(z) = c^2$.\\\\
Eisenstein triples are similar. Recall the Cosine rule (also known as the law of cosines)
\begin{equation*}
    c^2 = a^2 + b^2 - 2ab\cos(C)
\end{equation*}
and suppose we are looking for triangles with one angle of $60$ degrees. We know $\cos(60) = \frac{1}{2}$ and thus the equation becomes
\begin{equation*}
    c^2 = a^2 - ab + b^2
\end{equation*}
and triples that satisfy such equation (look familiar?) are called \textit{Eisenstein triples}.\\\\
We now introduce Eisenstein integers. These are to Eisenstein triples what Gaussian integers are to Pythagorean triples. Eisenstein integers are the ring
\begin{equation*}
    \Z[\omega] = \{a + b\omega : a, b \in \Z\}
\end{equation*}
where $\omega$ is a cube root of unity. Being a cube root of unity gives $\omega$ a lot of properties, so we explore it in its entirety. By definition, a cube root of unity is a root to the equation
\begin{equation*}
    (\omega-1)(\omega^2 + \omega + 1) = 0
\end{equation*}
and thus we can assert our first property $\omega^2 = -(1 + \omega)$. We often represent (the complex values of) $\omega$ 
\begin{equation*}
    \omega = \frac{-1 \pm i\sqrt{3}}{2} = e^{\frac{2\pi i }{3}}
\end{equation*}
and it's easy to show that $\conj{\omega} = \omega^2$ (consider the transformation on an Argand diagram).\\\\
Circling back to our Diophantine equation; how are these linked? We've mentioned that they are analogous to that of Gaussian integers to the Pythagorean triples and indeed, if we consider the norm of some $z \in \Z[\omega]$
\begin{equation*}
    N(z) = z\conj{z} = (a+b\omega)(a+b\conj{\omega}) = \dots = a^2 - ab + b^2
\end{equation*}
where the algebraic simplification is an exercise in using the properties of $\omega$ outlined above. So, as we saw before, we can generate an Eisenstein triple $(a, b, c)$ by considering the norm of an Eisenstein integer $N(a + b\omega)$ \dots\\\\
\dots and we've seen how this is done by Johnson \cite{johnson2011} and Read \cite{read2006}. They have different approaches and formulas but one is given (in this case by Read)
\begin{quote}
    \textit{Let $p, q$ be positive integers such that $p$ and $q$ are coprime and $q$ is not divisible by $3$. Then the triple
    \begin{align*}
        a &= 2pq + q^2,\\
        b &= 3p^2 + 4pq + q^2,\\
        c &= 3p^2 + 3pq + q^2
    \end{align*}
    form a (primitive) triangle such that one of the angles is 60 degrees}
\end{quote}
\dots i.e an Eisenstein triple. We can test this with $p = 1, q = 2$ (for example) and see that we indeed have a solution $a = 8, b = 15, c= 13$ where the cubic $x(x-8)(x-15)$ is ``nice". This is not exhaustive; there are other formulae (at least for Read, Johnson is slightly more complicated) needed.
\\
Here is a non-exhaustive list of Eisenstein triples with $a$, $b$, and $c$ as previously defined.
\begin{center}
\begin{tabular}{||c c c||} 
 \hline
 a & b & c \\ [0.5ex] 
 \hline\hline
 3 & 8 & 7 \\ 
 \hline
 5 & 8 & 7 \\
 \hline
 5 & 21 & 19 \\
 \hline
 7 & 15 & 13 \\
 \hline
 7 & 40 & 37 \\
 \hline
 8 & 15 & 13 \\
 \hline
 9 & 24 & 21 \\
 \hline
 11 & 35 & 31 \\
 \hline
 11 & 96 & 91 \\
 \hline
 16 & 55 & 49 \\
 \hline
 31 & 255 & 241 \\ [1ex] 
 \hline
\end{tabular}
\end{center}

\subsection{Can We Use It?}
If we can find a new way to generate triples/solve this Diophantine equation (perhaps using Eisenstein triples, perhaps using complex numbers) then we can use it in our project. If not, Chris probably won't approve - it's ``nothing new". At the very least it can be used in common themes - this is the second time Eisenstein triples have come up now, they're also seen in Cuoco's paper \autocite{cuoco2000} and applied to the context of finding nice cubics.

\subsection{Lara's Conjecture \& Co}
After trying for a long time to work with the most general form of cubic, $$(x+a)(x+b)(x+c),$$ I was struggling to get to a reasonable equation for conditions to produce two integer turning points. I began to wonder if fixing a root would make the resulting maths less complicated and possibly produce a nicer condition. After fixing $$(x-1)(x+a)(x+b),$$ expanding, differentiating and applying the quadratic formula to the resulting quadratic, I noticed that if $a=b$ then our discriminant for the x-value of the turning points would become $(a+1)^2$: a perfect square. From this I manipulated the two possible values of $x$ to get equations for the value of $a$ in terms of any natural number $k$ and found that one was always an integer and the other produced a very simple equation. The conjecture and proof are now as follows.\\\\
\textbf{Conjecture:} If a cubic is of the form $(x-1)(x+a)^2$ then it has two \textit{integer turning points} when $a = 2-3k$ for for all $k \in \mathbb{Z}$\\

\textbf{Proof:} Let $f(x)=(x-1)(x+a)^2$, expanding, we get that $$f(x)=x^3+(2a-1)x^2+(a^2-2a)x-a^2$$ 
Taking the first derivative we obtain the following:
$$f'(x)=3x^2+(4a-2)x+a^2-2a$$
Using the quadratic formula, we discover that the turning points of $f(x)$ occur when $$x=\frac{2 - 4a \pm \sqrt{(4a - 2)^2 - 12(a^2 - 2a)}}{6}$$
$$=\frac{2 - 4a \pm \sqrt{16a^2-16a+4-12a^2+24a}}{6}$$
$$=\frac{2 - 4a \pm \sqrt{4(a^2+2a+1}}{6}$$
$$=\frac{1 - 2a \pm \sqrt{(a+1)^2}}{3}$$
$$=\frac{1 - 2a \pm a+1}{3}$$
Therefore, we get that 
$x=\frac{2 - a}{3}$ or $x=-a$\\
Hence, in order for our turning points to be at a integer values of $x$, it must be true that $a=2-3k$ for any $k \in \mathbb{Z}$ as required.\\\\
I then tried the same thing with a fixed root at $-1$ and, unsurprisingly, found that a very similar outcome was produced.\\
\textbf{Conjecture:} If a cubic is of the form $(x+1)(x+a)^2$ then it has two \textit{integer turning points} when $a = -2-3k$ for for all $k \in \mathbb{Z}$. The proof for this conjecture follows from the last.\\\\


\textbf{Conjecture:} If a cubic is of the form $x^3 + bx^2 + cx + d$ where $b$ and $c$ are both multiples of $3$, then it will have \textit{integer turning points} if $k_1^2 - k_2$ is a perfect square, i.e. $\sqrt{k_1^2 - k_2}$ is an integer, for $b = 3k_1$ and $c = 3k_2$. \\

\textbf{Proof:} Let $g(x) = x^3 + bx^2 + cx + d$ where $b = 3k_1$ and $c = 3k_2$ with $k_1, k_2 \in \mathbb{Z}$. Taking the first derivative, we obtain:
$$g'(x) = 3x_2 + 2bx + c$$
Applying the quadratic formula, this leads to turning points of $g(x)$ occurring when
\begin{align*}
    x &= \frac{-2b \pm \sqrt{4b^2 - 12c}}{6} \\
    &= \frac{-b \pm \sqrt{b^2 - 3c}}{3} \\
    &= \frac{-3k_1 \pm \sqrt{9k_1^2 - 9k_2}}{3} \\
    &= \frac{-3k_1 \pm 3 \sqrt{k_1^2 - k_2}}{3} \\
    &= -k_1 \pm \sqrt{k_1^2 - k_2}
\end{align*}
Since $k_1$ is an integer, for $x$ to be an integer, $\sqrt{k_1^2 - k_2}$ needs also be an integer. Thus, for $g(x)$ to have integer turning points, $k_1^2 - k_2$ must be a perfect square.\\\\

I also wanted to see if I could come to any nice conclusions considering cubics that don't have three real roots. At first I tried the obvious, general form $(x+c)(x+(a+bi))(x+(a-bi))$, expanded, differentiated, and plugged into the quadratic formula. This led to obtaining a discriminant of $(a-c)^2-3b^2$ meaning for the easiest option, a discriminant equal to $0$, there was a requirement of $\frac{(a-c)^2}{b^2}=3$. At this point, I saw that making $b=1$ leaves me with a more simple equation to solve.\\
Restarting my attempts, I tried the equation $(x+a)(x+(b+i))(x+(b-i))$ and followed the same procedure until finding the discriminant to be $(a-b)^2-3$. I noticed that in order for this to be a perfect square, our only possible condition is $(a-b)^2=4$ as $1$ and $4$ are the only square numbers with a difference of $3$, therefore our first condition must be that $a-b=\pm2$. From this, I obtained the following.\\\\
\textbf{Conjecture:} A cubic of the form $(x+a)(x+(b+i))(x+(b-i))$ where $a=b+2$ has exactly one integer value of $x=-1-b$ as a turning point and exactly one rational value of $x=\frac{-3b-1}{3}$ as the other turning point.\\\\
\textbf{Proof:} Let $f(x)=(x+a)(x+(b+i))(x+(b-i))$. As $a=b+2$, we will substitute this into our original equation to obtain 
$$f(x)=(x+b+2)(x+(b+i))(x+(b-i)).$$
Expanding, we get that:
$$f(x)=(x+b+2)(x^2+2bx+b^2+1)$$
$$=x^3+2bx^2+b^2x+x+bx^2+2b^2x+b^3+b+2x^2+4bx+2b^2+2$$
$$=x^3+(3b+2)x^2+(3b^2+4b+1)x+(b^3+2b^2+b+2).$$
Differentiating this, we obtain
$$f'(x)=3x^2+(6b+4)x+3b^2+4b+1$$
Applying the quadratic formula to find the points at which the first derivative equals $0$, we find that:
$$x=\frac{-(6b+4)\pm\sqrt{(6b+4)^2-(4*3*(3b^2+4b+1))}}{2*3}$$
$$x=\frac{-6b-4\pm\sqrt{36b^2+48b+16-36b^2-48b-12}}{6}$$
$$=\frac{-6b-4\pm\sqrt{4}}{6}$$
$$=\frac{-6b-4\pm2}{6}$$
Therefore, either
$$x=\frac{-6b-6}{6}=-b-1$$ or $$x=\frac{-6b-2}{6}=\frac{-3b-1}{3}$$\\
Another nice result of this, that should seem fairly obvious, is that for the integer value of $x$, our $y$ co-ordinate will also be an integer. For our rational value of $x$, the highest value the denominator of $y$ can take in its lowest rational form is $27$.\\\\

\textbf{Conjecture:} A cubic of the form $(x+a)(x+(b+i))(x+(b-i))$ where $a=b-2$ has exactly one integer value of $x=1-b$ as a turning point and exactly one rational value of $x=\frac{-3b+1}{3}$ as the other turning point. The proof for this conjecture follows from the last.\\\\


\subsection{With Depressed Cubics}
What if we loosen up the criteria needed for ``nice" cubics? For example, suppose a teacher is setting problems on finding only the turning point of a cubic equation; we do not care for the roots. Consider a depressed cubic $P(t) = t^3 + pt + q$. Then its derivative satisfies
\begin{equation*}
    P'(t) = 0 \implies t = \frac{\pm \sqrt{-12p}}{6} = \frac{2\sqrt{-3p}}{6}.
\end{equation*}
Thus, in order for $t$ to an integer, we require $-3p = n^2$ for some integer $n \in \N_+$. With the restriction that $p$ is an integer itself, we need $-3p = m^2$ for some integer $m \in \{3n : n \in \N\}$ i.e
\begin{equation*}
    p \in \{-3, -12, -27, -48, \dots\}.
\end{equation*}
Using python, we can generate these ``nice" cubics. Below are some examples where $q$ is any integer.
\begin{center}
\begin{tabular}{c|c}
Cubic (in $t$) & Turning Points\\
\hline 
$t^3 - 3t + q$ & $(-1, q + 2), (1, q - 2)$\\
$t^3 - 12t + q$ & $(-2, q + 16), (2, q - 16)$\\
$t^3 - 48t + q$ & $(-3, q + 54), (3, q - 54)$
\end{tabular}
\end{center}

\noindent \href{https://www.desmos.com/calculator/harieviric}{Here is an interactive Desmos link} that plots these curves. We could easily modify the criteria for this section such that we can generate cubics with a purely \textit{rational} turning point - indeed, taking $p$ such $-3p = m^2$ for $m \in \N$ would be enough.

\subsection{With Vieta's Formulas}
\hl{To talk about how we can apply Vieta's formula again here.}


\section{Past Paper Questions}

\subsection{A-Level Further Maths Cubic Questions}
\textbf{Pearson Edexcel Level 3 GCE Further Mathematics
Advanced PAPER 1: Core Pure Mathematics 1}\\\\
\textbf{Question 1 June 2022:}\\\\
Consider the function 
\[f(z) = z^3 + az + 52\]
where \( a \) is a real constant. Given that \( 2 - 3i \) is a root of the equation \( f(z) = 0 \).
\begin{enumerate}
    \item[(a)] Write down the other complex root.
    \item[(b)] Hence,
    \begin{enumerate}
        \item[(i)] Solve completely \( f(z) = 0 \).
        \item[(ii)] Determine the value of \( a \).
    \end{enumerate}
    \item[(c)] Show all the roots of the equation \( f(z) = 0 \) on a single Argand diagram.\\\\
\end{enumerate}

\textbf{Question 3 June 2021:}\\\\
The cubic equation
\[ax^3 + bx^2 - 19x - b = 0\]
where \( a \) and \( b \) are constants, has roots \( \alpha \), \( \beta \), and \( \gamma \).

The cubic equation
\[w^3 - 9w^2 - 97w + c = 0\]
where \( c \) is a constant, has roots \( (4\alpha - 1) \), \( (4\beta - 1) \), and \( (4\gamma - 1) \).
Without solving either cubic equation, determine the value of \( a \), the value of \( b \), and the value of \( c \).\\\\

\textbf{Question 1 June 2020:}
Let 
\[f(z) = 3z^3 + pz^2 + 57z + q\]
where \( p \) and \( q \) are real constants. Given that \( 3 - 2\sqrt{2}i \) is a root of the equation \( f(z) = 0 \):
\begin{enumerate}
    \item[(a)] Show all the roots of \( f(z) = 0 \) on a single Argand diagram.    
    \item[(b)] Find the value of \( p \) and the value of \( q \).\\\\
\end{enumerate}

\textbf{Question 1 June 2019 (4th power not 3rd):}
\[f(z) = z^4 + az^3 + bz^2 + cz + d\] where \( a \), \( b \), \( c \), and \( d \) are real constants.\\
Given that \( -1 + 2i \) and \( 3 - i \) are two roots of the equation \( f(z) = 0 \):

\begin{enumerate}
    \item[(a)] Show all the roots of \( f(z) = 0 \) on a single Argand diagram.
    \item[(b)] Find the values of \( a \), \( b \), \( c \), and \( d \).\\\\
\end{enumerate}

\textbf{Pearson Edexcel Level 3 GCE Further Mathematics
Advanced PAPER 2: Core Pure Mathematics 2}\\\\

\textbf{Question 6 June 2022:}\\
The cubic equation
\[4x^3 + px^2 - 14x + q = 0\]
where \( p \) and \( q \) are real positive constants, has roots \( \alpha \), \( \beta \), and \( \gamma \).

Given that \( \alpha^2 + \beta^2 + \gamma^2 = 16 \):

\begin{enumerate}
    \item[(a)] Show that \( p = 12 \).
    
    Given that 
    \[\frac{1}{\alpha} + \frac{1}{\beta} + \frac{1}{\gamma} = \frac{14}{\alpha \beta \gamma}\]
    \item[(b)] Determine the value of \( q \).
    
    Without solving the cubic equation,
    \item[(c)] Determine the value of \( (\alpha - 1)(\beta - 1)(\gamma - 1) \).
\end{enumerate}

\textbf{Question 2 June 2019:}\\
The roots of the equation 
\[x^3 - 2x^2 + 4x - 5 = 0\]
are \( p \), \( q \), and \( r \).

Without solving the equation, find the value of:
\begin{enumerate}
    \item[(i)] \( 2p + q^2 + 2r \)
    \item[(ii)] \( (p - 4)(q - 4)(r - 4) \)
    \item[(iii)] \( p^3 + q^3 + r^3 \)
\end{enumerate}



\subsection{A-Level Maths Cubic Questions}

\textbf{Pearson Edexcel Level 3 GCE Mathematics
Advanced PAPER 1: Pure Mathematics 1}\\\\
\textbf{Question 6 June 2022:}\\
A curve \( C \) has equation \( y = f(x) \), where \( f(x) \) is a cubic expression in \( x \). The curve:
\begin{itemize}
    \item passes through the origin,
    \item has a maximum turning point at \( (2, 8) \),
    \item has a minimum turning point at \( (6, 0) \).
\end{itemize}
\begin{enumerate}
    \item[(a)] Write down the set of values of \( x \) for which \( f'(x) < 0 \).    
    \item[(b)] The line with equation \( y = k \), where \( k \) is a constant, intersects \( C \) at only one point. Find the set of values of \( k \), giving your answer in set notation.    
    \item[(c)] Find the equation of \( C \). You may leave your answer in factorised form.
\end{enumerate}


\textbf{Pearson Edexcel Level 3 GCE Mathematics
Advanced PAPER 2: Pure Mathematics 2}\\\\
\textbf{Question 1 June 2023:}\\
Let \[f(x) = x^3 + 2x^2 - 8x + 5.\]
\begin{enumerate}
    \item[(a)] Find \( f''(x) \). 
    \item[(b)] 
    \begin{enumerate}
        \item[(i)] Solve \( f''(x) = 0 \).
        \item[(ii)] Hence, find the range of values of \( x \) for which \( f(x) \) is concave.\\\\
    \end{enumerate}
\end{enumerate}

\textbf{Question 5 June 2023:}\\
The curve \( C \) has equation \( y = f(x) \). The curve
\begin{itemize}
    \item passes through the point \( P(3, -10) \)
    \item has a turning point at \( P \)
\end{itemize}

Given that 
\[\frac{dy}{dx} = 2x^3 - 9x^2 + 5x + k\]
where \( k \) is a constant,
\begin{enumerate}
    \item[(a)] Show that \( k = 12 \).
    \item[(b)] Hence, find the coordinates of the point where \( C \) crosses the y-axis.\\\\
\end{enumerate}

\textbf{Question 7 June 2023:}\\
A curve has equation
\[x^3 + 2xy + 3y^2 = 47.\]
\begin{enumerate}
    \item[(a)] Find \( \frac{dy}{dx} \) in terms of \( x \) and \( y \). 
    The point \( P(-2, 5) \) lies on the curve.    
    \item[(b)] Find the equation of the normal to the curve at \( P \), giving your answer in the form \( ax + by + c = 0 \), where \( a \), \( b \), and \( c \) are integers to be found.\\\\
\end{enumerate}

\textbf{Question 5 November 2021:}\\
The curve \( C \) has equation 
\[y = 5x^4 - 24x^3 + 42x^2 - 32x + 11, \quad x \in \mathbb{R}\]
\begin{enumerate}
    \item[(a)] Find:
    \begin{enumerate}
        \item[(i)] \( \frac{dy}{dx} \)
        \item[(ii)] \( \frac{d^2y}{dx^2} \)
    \end{enumerate}  
    \item[(b)] Verify that:
    \begin{enumerate}
        \item[(i)] \( C \) has a stationary point at \( x = 1 \).
        \item[(ii)] Show that this stationary point is a point of inflection, giving reasons for your answer.
    \end{enumerate}    
\end{enumerate}


\section{Project Plan}
Things to include in a possible ``introduction"
\begin{enumerate}[itemsep=-2mm]
    \item What is meta-mathematics? What are ``nice" problems?
    \item What is the history of meta-mathematics? Where have we seen it before? 
    \item What are we focusing on in this paper? What is the goal of the paper?
    \item Outline the next sections?
\end{enumerate}
We're focusing on the cubic, so we could dedicate a whole section to it. This could include \hl{(not a conclusive or exhaustive list! just throwing ideas out)}
\begin{enumerate}[itemsep=-2mm]
    \item What is the cubic? Why is it interesting? In what context do we want meta-mathematics to help us? (Think exam papers?)
    \item Some not-so elementary results about cubics (e.g depressed cubics)
    \item ``Nice" cubics in the sense of integer roots \textit{and} integer turning points.
    \item ``Nice" cubics in the sense of integer roots \textit{and} rational turning points.
    \item ``Nice cubics" in the sense of just integer turning points.
    \item ``Nice cubics in the sense of just rational turning points.
    \item Cardano's formula - if we want?
    \item Our own conjectures and their proofs (on cubics)
    \item Rational integrals \hl{(* this is new - to discuss)}
\end{enumerate}
and during this section we would surely talk about some common themes that would could explore in more detail in a further chapter;
\begin{enumerate}[itemsep=-2mm]
    \item Pythagorean triples - how to generate them and some examples (coded).
    \item Eisenstein triples (and their link to the Eisenstein integers) and how to generate them, with examples (coded).
    \item Diophantine equations \hl{(would need to do more research on these, but they come up a few times).}
    \item ``By a suitable transformation".
\end{enumerate}

\noindent \hl{A big question is:} do we have enough material on cubics alone? Or do we need to investigate another ``branch" of meta mathematics?\\\\
\textbf{Notes from Chris meeting 06/11.}\\
Analysis vs synthesis. History of meta mathematics as a systematic review. History of can include Plimpton 322. Complex numbers and their uses (Eisenstein, etc). 



% End of document - with bibliography.
\pagebreak 


\nocite{*}
\printbibliography{}



\end{document}


